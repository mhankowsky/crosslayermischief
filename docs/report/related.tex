\section{Related Work}
There are a number of academic publications that we've referenced to gain a better understanding of the space we've developed in.  The topic of cybersecurity has been examined in the following fields and we're going to go into more depth examining these works.  We've looked at the general fields of: Networked Control Systems, Cyber-Physical Security, and Cross-Layer Mischief. 
\subsection{Networked Control Systems}
The academic community has developed a number of research papers on the topic of security in networked control systems.  However, there are two that are most relevant to learn from for our work.  The survey of security in wireless sensor networks and the examination of adversary models for networked control systems. \cite{chen,cardenas}
\subsection{Cyber-Physical Security}
Ultimately, our work falls under the blanket of the concept of Cyber-Physical Security.  This general topic has been explored more thoroughly by the academic community and there are many solutions that have been proposed to solved the problems the current model of security presents \cite{mclaughlin, beitollahi, mo}.  Specifically relevant to our concept of developing a MATLAB process driven network testbed is the work that has been done in process control security and the simulation of the Tennesee Eastman challenge.\cite{hashimoto, ricker}  there has also been work done on developing various testbeds for the evaluation of cyber-physical systems.\cite{siaterlis}   
\subsection{Cross-Layer Mischief}
Our further focus on the affect of cross-layer attacks within networked control systems has yielded another large body of work to reference.  Additionally, there has been work done to examine Networked control systems under these attacks.  A number of attacks have been examined and surveyed on these systems, including DoS attacks in general\cite{amin} and more specifically in ad hoc networks \cite{radosavac}, service authentication broken by timing attacks\cite{hayes}, and replay attacks \cite{mosinopoli}