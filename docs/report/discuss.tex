\section{Discussion}

The project was relatively successful with the completion of the simulation framework and the implementation of some basic attacks. While we would have liked to include more attacks and improved the framework to more accurately model modern SCADA networks, the construction of the communication bridges and basic example of a network was a significant project and can be expanded to larger models and simulations if needed. The pipe and bridge system can used used for arbitrary MATLAB or OMNeT++ models, thus our example of a distributed SCADA network controlling a Tennessee-Eastman process is just one example. The framework was constructed to be dynamic, thus allowing for plug and play or various network and MATLAB models.

In the course of development, several challenges impeded work on the framework. Working through C++'s abandonment of the standard cast and void* generic type from C was somewhat frustrating, meaning converting floats became difficult, as well as dealing with development for both 32 and 64-bit machines. Aside from these minor technical issues, some of the biggest problems arose from the way in which INET and OMNeT++ model data transfer. As mentioned about, OMNeT++ and INET only attempt to create a timing-accurate model for various network protocols and hardware. Data-transfer is relatively absent and thus, when sending packets across the network model containing data from either the sensors or controller, the actual values being sent where often lost. Thus, ensuring data was able to get through the network was a significant challenge as the INET framework is very large and finding the few lines which destroy and recreate the packet was difficult. While we did get the simulation running, INET should be capable of preserving objects across the network, and it is a mystery why it does not.

While we did not model an exact SCADA system with industry software, the dangers illustrated in our results are still of significant concern. Throwing the process into an unstable state was not difficult and control system in general require real-time control. Delaying or spoofing signals in such systems can lead to instability. In the case of nuclear power or other chemical processes, these instabilities can be disastrous. As more and more control systems find their way onto the Internet, remote attacks become easier and physical access to the plant or system becomes unnecessary. Attackers from across the country or even other nations can target and disrupt control systems for critical infrastructure. Thus, a significant threat exists and more work needs to be done to ensure the safety and security of the various critical systems now on the Internet.